\section{Función de las 3 Cadenas}

\subsection{Caminata Aleatoria Simple}

La caminata aleatoria simple implementada corresponde a un proceso en una estructura cíclica de n estados, donde las probabilidades de transición están definidas por p (probabilidad de avanzar al siguiente estado) y q = 1-p (probabilidad de retroceder al estado anterior). La matriz de transición P presenta la forma donde cada estado i puede transitar al estado (i+1) mod n con probabilidad p, o al estado (i-1) mod n con probabilidad q.

La implementación se realiza mediante la función \texttt{generar\_caminata\_aleatoria(n, p)} que construye la matriz de transición correspondiente:

\begin{lstlisting}[language=Python]
def generar_caminata_aleatoria(n, p):
    """
    Genera matriz de transición para caminata aleatoria cíclica.

    Args:
        n: Número de estados
        p: Probabilidad de ir al siguiente estado (q = 1-p al anterior)

    Returns:
        P: Matriz de transición
    """
    q = 1 - p
    P = np.zeros((n, n))

    for i in range(n):
        P[i, (i + 1) % n] = p  # Probabilidad de ir al siguiente (mod n para ciclo)
        P[i, (i - 1) % n] = q  # Probabilidad de ir al anterior (mod n para ciclo)

    return P
\end{lstlisting}

La configuración experimental utiliza un rango de estados desde n = 10 hasta n = 560 con incrementos de 50, y cuatro valores de probabilidad p = [0.2, 0.4, 0.6, 0.8]. Para cada combinación de parámetros, se genera la matriz de transición correspondiente y se aplican ambos métodos de cálculo de distribución estacionaria, midiendo los tiempos de ejecución mediante \texttt{perf\_counter()} y calculando el error de convergencia entre los resultados obtenidos por ambos métodos.

\section{Bloques de Código que Generan los Cambios en las Cadenas}

% Contenido a desarrollar: implementación y modificaciones aplicadas a las cadenas